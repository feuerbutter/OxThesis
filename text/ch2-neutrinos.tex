\begin{savequote}[8cm]
Alles Gescheite ist schon gedacht worden.\\
Man muss nur versuchen, es noch einmal zu denken.

All intelligent thoughts have already been thought;\\
what is necessary is only to try to think them again.
  \qauthor{--- Johann Wolfgang von Goethe \cite{von_goethe_wilhelm_1829}}

Dancing is more important. --- Pauli
\end{savequote}

\chapter{\label{ch:2-neutrinos}The Field of Neutrinos}

\minitoc

\section{The Past}
\section{The Present}
\section{The Plans}


This document introduction won't serve as a complete primer on \LaTeX.  There are plenty of those online, and googling your questions will often get you answers, especially from \url{http://tex.stackexchange.com}.

Instead, let's talk a little about a few of the features and packages lumped into this template situation.  The \verb|savequote| environment at the beginning of chapters can add some wittiness to your thesis.  If you don't like the quotes, just remove that block.

For when it comes time to do corrections, there are two useful commands here.  First, the \verb|mccorrect| command allows you to highlight a short correction \mccorrect{like this one}.  When the thesis is typeset normally, the correction will just appear as part of the text.  However, when you declare \verb|\correctionstrue| in the main \verb|Oxford_Thesis.tex| file, that correction will be highlighted in blue.  That might be useful for submitting a post-viva, corrected copy to your examiners so they can quickly verify you've completed the task.

\begin{mccorrection}
For larger chunks, like this paragraph or indeed entire figures, you can use the \verb|mccorrection| environment.  This environment highlights paragraph-sized and larger blocks with the same blue colour.
\end{mccorrection}
