\begin{savequote}[8cm]
Alles Gescheite ist schon gedacht worden.\\
Man muss nur versuchen, es noch einmal zu denken.

All intelligent thoughts have already been thought;\\
what is necessary is only to try to think them again.
  \qauthor{--- Johann Wolfgang von Goethe \cite{von_goethe_wilhelm_1829}}

Dancing is more important. --- Pauli
\end{savequote}

\chapter{\label{ch:2-neutrinos}The Field of Neutrinos}
\minitoc

As a neutrino physicist, an overall understanding of the field is necessary.

\section{The History}
\subsection{The Past}
The field of neutrino has a dramatic opening that many know of - in a letter~\cite{Pauli:1930pc} from Wolfgang Pauli to the Radioactivity conference in 1930, which he did not attend in person as he was rumoured to be going for a dance instead. 
The letter was proposing a solution to the then very puzzling problem of $\beta$ decay.
It was thought of a two body decay, where the heavier nucleus decay to a lighter nucleus and an electron, which carries away a well-definied amount of energy due to the mass difference between the parent and daughter nucleus.  
However, the measured electron energy spectrum was a wide distribution rather than the fixed value expected. 
In his letter, Pauli postulated a third but un-detected particle in the decay that shared the energy budget with electron so that the sum of the third particle energy and the electron energy is equal to the fixed value expected, but not the electron energy alone. 
Originally, he coined this particle the ``neutron'', but this name did not stick and was given to the now known neutron discovered by James Chadwick in 1932.
The present name, ``neutrino'', was later coined by Edoardo Amaldi in a conversation with Enrico Fermi, who later published the seminal paper on his theory of beta decay in 1934, laying the solid theoretical foundation for subsequent experimental work.
However, the cross section for neutrino interaction is estimated to be so small and hence extremely difficult to observe that even Pauli lamented, 
``I have done a terrible thing. I have postulated a particle that cannot be detected.''

It was such a formidable task that Frederick Reines even proposed to use the atomic bomb  as the neutrino source to have an intense source of neutrino for detection.
Fortunately, Reines and his collaborator, Clyde Cowan, discovered that the delay of the signal of the neutron capture from the signal of the annihilation between electron and positron, the product of the anti-neutrino interaction, is distinct and could be utilized to significantly reduce backgrounds such that a less intense (dangerous) source of (anti)neutrino can be used.
It was not until about 26 years after the postulation of neutrino that it was directly detected in the Cowan-Reines experiment in 1956. 
This remarkable achievement was recognised with a Nobel Prize in 1995.

To better appreciate the historical progress of neutrino physics, it is clearer to put it in the perspective of the overall development of particle physics.

It was suggested by Markov [65], Pontecorvo [66], and Schwartz [67] to use proton accelerators to produce high energy
neutrino beam from pion decays to perform experiments like:
nu+n->mu+n
etc
The experiments performed at the Brookhaven National Laboratory (BNL) by Danby et al. [71] and later at
CERN by Bienlein et al. [72] 
observed only muon but never elec tron, confirming numu and nue are different (1962)

1965 BNL introducted the POT idea as a measure of the neutrino flux

Tau in 1975
the existence of a new flavor of
neutrinos ντ was proposed, which was observed much later in the DONUT experiment [74,75] in 2000 at the Fermilab

It was Fermi [3,4] and Perrin [5] who first discussed the determination of the neutrino mass from the study of the
end-point spectrum of beta decay.

Pontecorvo [99] in 1957 proposed the idea of neutrino oscillation by stating that the physical state of neutrinos
produced in weak interaction processes is a superposition of neutrino and antineutrino states with definite masses.
Later Maki, Nakagawa and Sakata [101] applied the idea of neutrino
oscillation in flavor space in which neutrino oscillation between neutrinos of two flavor i.e. νe and νµ was proposed
and was later extended to three flavors of neutrinos.


\subsection{The Present}
\subsection{The Plans}

\section{The Theory}
\subsection{Oscillation}
\subsection{Cross section}

This document introduction won't serve as a complete primer on \LaTeX.  There are plenty of those online, and googling your questions will often get you answers, especially from \url{http://tex.stackexchange.com}.

Instead, let's talk a little about a few of the features and packages lumped into this template situation.  The \verb|savequote| environment at the beginning of chapters can add some wittiness to your thesis.  If you don't like the quotes, just remove that block.

For when it comes time to do corrections, there are two useful commands here.  First, the \verb|mccorrect| command allows you to highlight a short correction \mccorrect{like this one}.  When the thesis is typeset normally, the correction will just appear as part of the text.  However, when you declare \verb|\correctionstrue| in the main \verb|Oxford_Thesis.tex| file, that correction will be highlighted in blue.  That might be useful for submitting a post-viva, corrected copy to your examiners so they can quickly verify you've completed the task.

\begin{mccorrection}
For larger chunks, like this paragraph or indeed entire figures, you can use the \verb|mccorrection| environment.  This environment highlights paragraph-sized and larger blocks with the same blue colour.
\end{mccorrection}
