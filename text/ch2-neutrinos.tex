\begin{savequote}[8cm]
Alles Gescheite ist schon gedacht worden.\\
Man muss nur versuchen, es noch einmal zu denken.

All intelligent thoughts have already been thought;\\
what is necessary is only to try to think them again.
  \qauthor{--- Johann Wolfgang von Goethe \cite{von_goethe_wilhelm_1829}}

Dancing is more important. --- Pauli
\end{savequote}

\chapter{\label{ch:2-neutrinos}The Field of Neutrinos}
\minitoc

As a neutrino physicist, an overall understanding of the field is necessary.

\section{The History}
\subsection{The Past}
The field of neutrino has a dramatic opening that many know of - in a letter~\cite{Pauli:1930pc} from Wolfgang Pauli to the Radioactivity conference in 1930, which he did not attend in person as he was rumoured to be going for a dance instead. 
The letter was proposing a solution to the then very puzzling problem of $\beta$ decay.
It was thought of a two body decay, where the heavier nucleus decay to a lighter nucleus and an electron, which carries away a well-definied amount of energy due to the mass difference between the parent and daughter nucleus.  
However, the measured electron energy spectrum was a wide distribution rather than the fixed value expected. 
In his letter, Pauli postulated a third but un-detected particle in the decay that shared the energy budget with electron so that the sum of the third particle energy and the electron energy is equal to the fixed value expected, but not the electron energy alone. 
Originally, he coined this particle the ``neutron'', but this name did not stick and was given to the now known neutron discovered by James Chadwick in 1932.
The present name, ``neutrino'', was later coined by Edoardo Amaldi in a conversation with Enrico Fermi, who later published the seminal paper on his theory of beta decay in 1934~\cite{Fermi:1934hr}, laying the solid theoretical foundation for subsequent experimental work.
However, the cross section for neutrino interaction is estimated to be so small and hence extremely difficult to observe that even Pauli lamented, 
``I have done a terrible thing. I have postulated a particle that cannot be detected.''

It was such a formidable task that Frederick Reines even proposed to use the atomic bomb  as the neutrino source to have an intense source of neutrino for detection.
Fortunately, Reines and his collaborator, Clyde Cowan, discovered that the delay of the signal of the neutron capture from the signal of the annihilation between electron and positron, the product of the anti-neutrino interaction, is distinct and could be utilized to significantly reduce backgrounds such that a less intense (dangerous) source of (anti)neutrino can be used.
It was not until about 26 years after the postulation of neutrino that it was directly detected in the Cowan-Reines experiment at the Savannah River plant in 1956~\cite{Cowan:1956rrn}. 
The detector was mainly composed of $1400$ litres of liquid scintillator and $200$ litres of Cadium-loaded water as target and neutron capturer.
This remarkable achievement was recognised with a Nobel Prize in 1995.

After the Cowan-Reines experiment, the existence of neutrion is beyond doubt and the field of neutrino phsyics developed rapidly. 

There were two naturally complementary directions for development.
One is to build larger and more advanced detectors to measure the existing neutrino sources, for example, the solar neutrinos, the reactor neutrinos, the atmospheric neutrinos and others.
The other is to include the neutrino production as part of the experiment, which becomes the field of accelerator neutrinos.

Roughly about the same time, Raymond Davis was also working in the Savannah River plant, using the inverse beta decay of chlorine to argon, suggested by Pontecorvo~\cite{Pontecorvo:1946mv}, to detect neutrinos.
His experiment measured a $20$ times smaller crosss section for anti-neutrino than for neutrino in the inverse beta decay turning cholrine into argon, providing evidence for neutrino antineutrino being different particles~\cite{Davis:1959pba}.
Later he applied the same technique to measure the solar neutrino flux with a detector of the size of $378,000$ litres at the Homestake Mine~\cite{Davis:1964zz}, sufficient to detect solar neutrinos based on the Standard Solar Model calculation done by John Bahcall~\cite{Bahcall:1964ya}. 
It was to everyone's surprise that the Homestake experiment only observed about $1/3$ of the expected flux~\cite{Davis:1968}, a discrepancy known as the solar neutrino problem.
Davis and Bahcall did meticulous checks on their results and found no errors large enough to explain the discrepancy.
This phenonemon has sparked many efforts, both experimentally and theoretically, to resolve the observed discrepancy.
The solution we now know consists of neutrino oscillation and the Mikheyev-Smirnov-Wolfenstein (MSW) effect~\cite{Wolfenstein:1977ue,Mikheyev:1985zog}.
The idea of neutrino oscillation was actually proposed even before the solar neutrino problem by Pontecorvo in 1957~\cite{Pontecorvo:1957qd} and completed by Maki, Nakagawa and Sakata in 1962~\cite{Maki:1962mu}.
However, it was just one of the possible solutions back then among many other theories.
Papers on the other piece of the solution, the MSW effect, was published in the 1980s.
The combined solution, the large-mixing-angle MSW solution, provided an attractive answer to the solar problem.
Subsequent experiments, such as Kamiokande, a $3,000,000$-litre water cherenkov experiment lead by Masatoshi Koshiba~\cite{Kamiokande-II:1989hkh}, GALLEX~\cite{GALLEX:1998kcz} and SAGE~\cite{SAGE:1999nng}, confirmed the solar neutrino deficit.
The definitive evidence came in 2001, when the Sudbury Neutrino Observatory (SNO) experiment, with a $1,000,000$-litre heavy water Cerenkov detector, published its result~\cite{SNO:2001kpb}, directly showing that the observed deficit of electron neutrino is oscillated into other flavours.
In the next year, Davis and Koshiba promptly received a Nobel Prize in Physics.
Meanwhile, in 1998, the upgrade of Kamiokande, the Super-Kamiokande ($50,000,000$-ton water Cerenkov detector), reported the observation of neutrino oscillation in atmospheric neutrinos~\cite{Super-Kamiokande:1998kpq}.
Takaaki Kajita (Super-Kamiokande) and Arthur McDonald (SNO) shared a Nobel prize for their work on neutrino oscillation in 2015.

Along the other development route, Pontecorvo suggested the possibility of using accelerated proton hitting on targets to produce mesons that decay to produce a beam of neutrinos in 1959~\cite{Pontecorvo:1959sn}.
A year later, a similar proposal was made independently by Mark Schwartz~\cite{Schwartz:1960hg} to use the accelerator-produced neutrinos to address another puzzle at that time, if the muon neutrino is the same as the electron neutrino.
Schwartz and his collaborators, which included Leon Lederman and Jack Steinberger, used the Alternate Gradient Synchrotron at the Brookhaven National Laborarory to perform the first accerlator neutrino beam. 
The neutrino interaction was recorded by a $10$-ton spark chamber, which were able to reconstruct clear muon tracks from the interaction.
In 1962, the collected data proved that muon neutrinos are distinct from electron neutrinos~\cite{Danby:1962nd}.
Schwartz, Lederman and Steinberger were awarded the Nobel Prize in 1988.
For the subsequent decades, the accelerator neutrinos were used in turn to help progress the more general field of particle physics due to the unique nature of neutrino scattering.
For instance, neutral current interaction was first observed by the Gargamelle bubble chamber at CERN using a beam of neutrinos~\cite{GargamelleNeutrino:1973jyy}.
High energy neutrino beams were also deployed to test the electroweak theory and study the nucleon structure by the CDHS experiment at CERN~\cite{Schlatter:2015nxk}.

With neutrino oscilaltion finally confirmed, the field of neutrino physics has a fruitful past and a clear theoretical framework laid out.
The field has entered a phase of precision measurement that will be illustrated in the next subsection.
The two complementary development routes in the past has combined in long-baseline experiments to elucidate the few remaining questions in the field.

% To better appreciate the historical progress of neutrino physics, it is clearer to put it in the perspective of the overall development of particle physics.

% Electron was discovered by J.J. Thomson in 1897.
% Proton was discovered by Rutherford in 1917.
% Neutron was discovered by Chadwick in 1932.
% Paul Dirac published the famous Dirac equation in 1928, predicting the existence of the positron, the anti-particle of the electron, which was discovered by Carl Anderson in 1932.
% Beta decay was discovered by Lise Meitner and Otto Hahn in 1938 to be a weak interaction process, where a neutron decays to a proton, an electron and an anti-neutrino.
% The muon was discovered by Carl Anderson in 1936, which was initially thought to be the Yukawa particle responsible for the strong nuclear force.

% ``Weak interaction'' is first coined by Enrico Fermi in 1933 to describe the force responsible for beta decay. <CHECK>
% On the theoretical front, Fermi proposed a theoretical solution to the beta decay in 1934.

% In 1954, CN Yang and Robert Mills published the seminal paper on non-abelian gauge theory, laying the foundation for the Standard Model.
% In 1960s and 1970s, rapid progress was made in the unifications of the electromagnetic and weak interactions, leading to the Glashow-Salam-Weinberg (GSW) model, who shared the Nobel Prize in Physics in 1979. 
% The discovery of the W and Z bosons at CERN in 1983.

% It was suggested by Markov [65], Pontecorvo [66], and Schwartz [67] to use proton accelerators to produce high energy
% neutrino beam from pion decays to perform experiments like:
% nu+n->mu+n
% etc
% The experiments performed at the Brookhaven National Laboratory (BNL) by Danby et al. [71] and later at
% CERN by Bienlein et al. [72] 
% observed only muon but never elec tron, confirming numu and nue are different (1962)

% 1965 BNL introducted the POT idea as a measure of the neutrino flux

% Tau in 1975
% the existence of a new flavor of
% neutrinos ντ was proposed, which was observed much later in the DONUT experiment [74,75] in 2000 at the Fermilab

% It was Fermi [3,4] and Perrin [5] who first discussed the determination of the neutrino mass from the study of the end-point spectrum of beta decay.

\subsection{The Present}

\subsection{The Plans}
the Deep Underground Neutrino Experiment (DUNE)~\cite{DUNE:2016hlj,DUNE:2015lol,DUNE:2016evb,DUNE:2016rla,DUNE:2021tad}  

\section{The Theory}
The main goal of neutrino experiments is to measure all neutrino properties, including the mass, mixing angles, and the CP violation parameter, $\dcp$. 
The mixing angles and $\dcp$ fully describes neutrino oscillation, while the mass is the reason why neutrino oscillation happens.
Hence, these parameters will be better illustrated in the following Sec.~\ref{subsec:oscillation} in the context of the theoretical framework of neutrino oscillation.
As only products from neutrino interaction are detected in experiments, to achieve the said goal, a good understanding of neutrino interaction is necessary.
Hence, a brief overview of neutrino interaction is given in Sec.~\ref{subsec:interaction}.

\subsection{Oscillation}
\label{subsec:oscillation}
Neutrinos come in three flavours: electron neutrino ($\nu_e$), muon neutrino ($\nu_\mu$), and tau neutrino ($\nu_\tau$).
The different flavours of neutrinos will only produce the corresponding lepton in weak interaction, so they are eigenstates of the weak interaction.
However, neutrinos propagate through space as mass eigenstates, $\nu_1$, $\nu_2$, and $\nu_3$.
The reason for neutrino oscillation to occur is two-fold.
One is beacuse the mass eigenstates do not have a simple one-to-one correspondence with the weak eigenstates. 
They are a superposition of the weak eigenstates and vice versa.
The other is beacuse these mass eigenstates have different masses.
Neutrinos are created as flavour eigenstates in weak interaction, but they propagate as a linear combination of mass eigenstates, which propagate with different phase velocities due to their different masses.
Hence, throughout the propagation, the linear combination of mass eigenstates is changing constantly, leading to a changing superposition of flavour eigenstates, and thus, neutrino oscillation.

To be more specific, the matrix describing the mixing of mass eigenstates to flavour eigenstates is the PMNS matrix.
It is conventionally parameterized as follows:

\begin{equation}
U_{\text{PMNS}} = 
\begin{pmatrix}
1 & 0 & 0 \\
0 & c_{23} & s_{23} \\
0 & -s_{23} & c_{23}
\end{pmatrix}
\begin{pmatrix}
c_{13} & 0 & s_{13} e^{-i\dcp} \\
0 & 1 & 0 \\
-s_{13} e^{i\dcp} & 0 & c_{13}
\end{pmatrix}
\begin{pmatrix}
c_{12} & s_{12} & 0 \\
-s_{12} & c_{12} & 0 \\
0 & 0 & 1
\end{pmatrix}
\end{equation}
where $c_{ij} = \cos\theta_{ij}$ and $s_{ij} = \sin\theta_{ij}$, and $\dcp$ is the CP violation phase.
The angles, $\theta_{ij}$, are the mixing angles. 
Using the PMNS matrix, the flavour eigenstates can be expressed in terms of the mass eigenstates as follows:
\begin{equation}
\begin{pmatrix}
\nu_e \\
\nu_\mu \\
\nu_\tau
\end{pmatrix}
=
U_{\text{PMNS}}
\begin{pmatrix}
\nu_1 \\
\nu_2 \\
\nu_3
\end{pmatrix}
\end{equation}
To keep a close focus on the development of the theoretical idea, detailed derivations are rendered to the Appendix~\ref{app:oscillation}, and only important results are included here for elaboration.
Due to the small neutrino masses, the mass eigenstate, with mass $m_i$, acquires a phase approximately of
\begin{equation}
  \ket{\nu_i(L)} = e^{-i \frac{m_i^2L}{2E}} \ket{\nu_i(0)},
\end{equation}
after travelling a distance $L$ with energy $E$.
Hence, the probability of a neutrino of flavour $\alpha$ at creation to be detected as a neutrino of flavour $\beta$ after travelling a distance of $L$ is given by
\begin{array}{rl}
  P_{(\alpha \to \beta)} &= \left|\braket{\nu_\beta | \nu_\alpha(L)} \right|^2 = \left| \sum_i U_{\alpha i} U^*_{\beta i} e^{-i \frac{m_i^2L}{2E}} \right|^2 \\
  &= \delta_{\alpha\beta} - 4 \sum_{i>j} \text{Re} \left( U_{\alpha i} U^*_{\beta i} U^*_{\alpha j} U_{\beta j} \right) \sin^2 \left( \frac{\Delta m^2_{ij} L}{4E} \right) \\
  &+ 2 \sum_{i>j} \text{Im} \left( U_{\alpha i} U^*_{\beta i} U^*_{\alpha j} U_{\beta j} \right) \sin \left( \frac{\Delta m^2_{ij} L}{2E} \right),
\end{array}
where $\delta{\alpha\beta}$ is the Kronecker delta, $U_{\alpha i}$ is the element of the PMNS matrix, and $\Delta m^2_{ij} = m_i^2 - m_j^2$.
This result shows that the oscillation probability depends on the mixing angle through the PMNS matrix elements and on the mass difference and neutrino energy through the sine terms.
For brevity, the dependence on the sine terms can be characterised by a phase angle,
\begin{equation}
  \label{eq:osc-phase}
  \Delta_{ij} = \frac{\Delta m^2_{ij} L}{2E} \approx 1.27 \frac{\Delta m^2_{ij} \text{eV}^2 L \text{km}}{E \text{GeV}},
\end{equation}
where the oscillation is maximal when $\Delta_{ij} = \pi / 2$.
These results can be used to better understand how the various parameters are measured in different types of neutrino experiments.
The different sensitivies of these experiments are due to the relatively large difference in the mixing angles and the mass differences.
The latest values for the neutrino parameters based on Ref.~\cite{Capozzi:2021fjo,ParticleDataGroup:2024cfk} are given in Table~\ref{tab:neutrino-parameters}, which shows that the magnitude of $\delta_{21}$ is much smaller than that of $\delta_{31}$.
Unlike the CKM matrix, where all mixing angles are small, only $\theta_{13}$ is small while the other two angles are considerably larger, suggesting significant mixing.

\begin{table}[h]
  \centering
  \begin{tabular}{c|c}
    Parameter & Value \\
    \hline
    \hline
    $\Delta m^2_{21}~(\ev^2)$ & $7.36^{+0.16}_{-0.15} \times 10^{-5}$ \\
    $|\Delta m^2_{31}|~(\ev^2)$ & $2.448^{+0.023}_{-0.031} \times 10^{-3}$ \\
    $\theta_{12}$ ($\deg$) & $33.40^{+0.80}_{0.82}$ \\
    $\theta_{23}$ ($\deg$)       & $42.4^{+1.0}_{0.9}$ \\
    $\theta_{13}$ ($\deg$)       & $8.59^{+0.13}_{0.12}$ \\
    $\dcp$ ($\deg$) & $223^{+32}_{-23}$   \\
    \hline
  \end{tabular}
  \caption{The latest values for the neutrino parameters.}
  \label{tab:neutrino-parameters}
\end{table}



\subsubsection{Atmospheric neutrinos}
  Atmospheric neutrinos are mostly muon neutrinos produced from decays of mesons produced by cosmic rays interacting with the atmosphere.
  The neutrino energy is typically of the order of $1~\gev$, while the distrance travelled has a range of $O(10^2)$ to $O(10^4)~\km$.
  Substituting these numbers into \Eq.~\ref{eq:osc-phase}, $\Delta_{23}$ is of the order of $O(10^(-2))$ to $O(1)$, while $\Delta_{21}$ is of the order of $O(10^(-3))$ to $O(10^(-1))$.
  Hence, oscillation is much more prominent in the $\nu_\mu \to \nu_\tau$ channel than in the $\nu_\mu \to \nu_e$ channel.
  This is why atmospheric neutrino experiments are sensitive to $\theta_{23}$ and $\Delta m^2_{32}$, which are sometimes referred to as the atmospheric mixing angle and mass difference, respectively.

\subsubsection{Reactor neutrinos}
  Reactor neutrinos are mostly electron antineutrinos produced from nuclear fission in the power plants.
  The neutrino energy is typically of the order of $1~\mev$.
  Unlike the atmospheric neutrino measurement, detectors can be placed at various locations to measure different parameters.
  Thus, it is easier to discuss reactor neutrino oscillation by introducing the oscillation length variable given as:
  \begin{equation}
    \label{eq:osc-length}
    L_{\text{osc}} = \frac{4\pi E}{\Delta m^2} = 2.48 \frac{E \text{MeV}}{\Delta m^2 \text{eV}^2} \text{m},
  \end{equation}
  which corresponds to one complete period of neutrino oscillation.
  Substituting the reactor neutrino energy into Eq.~\ref(eq:osc-length), one gets $L_{\text{osc}} = O(10^2)~\km$ for $\delmtwoo$ and $L_{\text{osc}} = O(1)~\km$ for $\delmthro$. 
  Hence, by placing detectors kilometers away from the power plants, reactor neutrino experiments offer the unique opportunity to measure $\tothr$ and $\delmthro$, which are sometimes referred to as the reactor mixing angle and mass difference, respectively.
  Additionally, placing detectors at around $O(100)~\km$ away from the power plants allows the measurement of $\delmtwoo$.

  \subsubsection{Solar neutrinos}
  Solar neutrinos are mostly electron neutrinos produced from nuclear fusion in the sun.
  The majority of solar neutrinos have energies below $1~\mev$, except those from the $^8$B decay, which have energies up to $15~\mev$.
  The distance travelled by solar neutrinos is of the order of $O(10^8)~\km$.
  Substituting these values into Eq.~\ref{eq:osc-length}, one gets $L_{\text{osc}} = O(10^4)~\m$ for $\delmtwoo$ and $L_{\text{osc}} = O(10^3)~\m$ for $\delmthro$.
  Both are much smaller than the average distance travelled, so to observe the clear oscillatory pattern like in other neutrino experiments, the exact distance travelled by the individual neutrino needs to be known with an incredibly high precision, which is beyond the current detection capability.
  Fortunately, the average oscillation result can still be measured by the total survival rate of the electron neutrinos.
  
  On top of the usual neutrino oscillation, there is another important phenonemon at play in the solar neutrino measurements, which is the Mihheev-Smirnov-Wolfenstein (MSW) effect~\cite{Wolfenstein:1977ue,Mikheyev:1985zog}.
  A full discussion on the MSW effect is beyond the scope of this thesis.
  Simply put, the MSW effect is the modification of the mixing parameters due to the presence of electrons in matter. 
  This modification can enhance or suppress the oscillation probability depending on the mass difference and the electron density.
  Approximately, the required electron density is proportional to the mass difference squared and inversely proportional to the neutrino energy.
  Hence, the lower the neutrino energy or the larger the mass difference, the higher the electron density is required to modify the oscillation probability.
  Due to the considerably larger $\delmthro$, for almost the whole energy range of solar neutrinos, the electron density in the Sun is not sufficient to modify the mixing between $\nu_1$ and $\nu_3$.
  As for the relatively smaller $\delmtwoo$, the electron density is high enough to significantly modify the mixing between $\nu_1$ and $\nu_2$ for the high energy neutrinos from the $^8$B decay.
  The resultant impact on these high energy electron neutrinos is an alteration of its composition of the mass eigenstates such that when they leave the surface of the Sun, they are mainly composed of $\nu_2$, a considerably larger fraction than the unmodified composition.
  The MSW effect is demonstrated when different experiments measure solar neutrinos with different energies and observe a different survival rate of electron neutrinos.
  The absolute survival rates are sensitive to $\totwo$ and $\tothr$ for the low energy region and the high energy region, respectively.


  







\subsection{Interaction}
\label{subsec:interaction}

This document introduction won't serve as a complete primer on \LaTeX.  There are plenty of those online, and googling your questions will often get you answers, especially from \url{http://tex.stackexchange.com}.

Instead, let's talk a little about a few of the features and packages lumped into this template situation.  The \verb|savequote| environment at the beginning of chapters can add some wittiness to your thesis.  If you don't like the quotes, just remove that block.

For when it comes time to do corrections, there are two useful commands here.  First, the \verb|mccorrect| command allows you to highlight a short correction \mccorrect{like this one}.  When the thesis is typeset normally, the correction will just appear as part of the text.  However, when you declare \verb|\correctionstrue| in the main \verb|Oxford_Thesis.tex| file, that correction will be highlighted in blue.  That might be useful for submitting a post-viva, corrected copy to your examiners so they can quickly verify you've completed the task.

\begin{mccorrection}
For larger chunks, like this paragraph or indeed entire figures, you can use the \verb|mccorrection| environment.  This environment highlights paragraph-sized and larger blocks with the same blue colour.
\end{mccorrection}
