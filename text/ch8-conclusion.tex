\begin{savequote}[8cm]
\textlatin{Neque porro quisquam est qui dolorem ipsum quia dolor sit amet, consectetur, adipisci velit...}

There is no one who loves pain itself, who seeks after it and wants to have it, simply because it is pain...
  \qauthor{--- Cicero's \textit{de Finibus Bonorum et Malorum}}
\end{savequote}

\chapter{\label{ch:1-concl}Conclusion} 

\minitoc

Back in Oct. 2022, I was on my way to Tokai to join the task force for the assembly of the SFGD.
We first built the outer box that would hold the cubes together and the first layer was installed on 28th Oct.
For about two months, We stacked the scintillator cubes layer by layer and labouriously align them each time a new layer was added.
As the number of layers grew, it became harder and harder to perform vertical alignment due to the growing friction between a larger number of cubes.
At first the vertical alignment were made using thin metal rods of about $10$~cm long, which soon became too short.
The original plan was to use fishing lines for the subsequent alignment, but it was immediately apparent that they were not strong enough to push through the cubes.
Fortunately, longer metal rods were purchased in time that we did not have to perform the impossible.
The last layer was added on xxx Dec just before the new year.
After a short break, we moved on to the next task - fibre insertion.
More people joined and there were enough people and the workflow was indepedent enough that we could have a clear division of work.
Some were responsible for pulling out the fishing lines that were holding the cubes together for layer installation, some were pushing the optical fibres through the holes where the fishing lines have been removed and some were maintaining a constant supply of fibres for the fibre-pushers.
The fibre insertion took about another two months.
Afer that, I returned to focus on my analysis work while the electronics were added and calibrated diligently.
The assembly was finally completed in Oct. 2023 and was ready to be put into the ND280 pit.
In June 2024, the whole ND280 upgrade was completed and took data for the first time.
The pictures of data event display shown in this thesis were collected during the June 2024 run.

It was a truely unique and fortunate experience for a doctoral student to assemble a detector and to see it at work.
Although it is a shame that there is not enough time for me to perform a full cross section measurment using the new data, my MC analysis and the preliminary data-MC comparison demonstrate the great physics potential of the upgraded ND280. 
It will be able to measure TKI and the novel COM variables with unprecedented precison.
My tuning project using existing TKI measuremnts already shows promising improvement in neutrino interaction modelling.
It will be thus exciting to see the progress in interaction modelling brought by future TKI and COM measurements of ND280.
Besides advancing SM physics, my HNL sensitivity study of the SFGD demonstrates that it has comparable HNL detection capability as the gaseous TPCs. 
Thus, the whole upgraded ND280 has a detection volume almost twice as before.
Together with the to-be-upgraded beamline, T2K has the potential of making the most sensitive search for HNL produced from kaon and pion decay.

New and upgraded experiments, such as JUNO and SBND, across the world have also started operation or are about to.
It is optimistic that we could gain sufficient understanding of neutrino interaction that the next-generation LBL experiments can make a precise measurement of $\dcp$ within our generation.
I hope that my work could help to shorten the wait, even just a tiny bit, and when that day comes, I will be happy.