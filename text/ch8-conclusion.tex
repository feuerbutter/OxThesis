\begin{savequote}[8cm]
\begin{CJK*}{UTF8}{gbsn}
回首向来萧瑟处,
归去,
也无风雨也无晴。
\end{CJK*}

After all that has been through, at the end of the day, nothing matters, and everything will be alright.

  \qauthor{--- Shi Su \textit{Ding Feng Bo}}
\end{savequote}

\chapter{\label{ch:concl}Epilogue} 

\minitoc

Back in October 2022, I traveled to Tokai to join the task force responsible for assembling the SFGD.
We began by constructing the outer box to hold the cubes together, and the first layer was installed on October 28th.
For approximately two months, we meticulously stacked the scintillator cubes layer by layer, aligning each new layer with great effort.
As the number of layers increased, maintaining vertical alignment became progressively more challenging due to the growing friction between the accumulating cubes.
Initially, vertical alignment was achieved using thin metal rods approximately 10 cm in length, which soon proved too short.
The original plan was to utilize fishing lines for subsequent alignments; however, it quickly became evident that they lacked the necessary strength to shift the cubes in alignment.
Fortunately, we were able to purchase longer metal rods in time, preventing the need to undertake an impossible task.
The final layer was added in December just before the new year.

After a short break, we proceeded to the next task: fiber insertion.
With more team members joining, the workforce became sufficiently large and the workflow sufficiently independent to allow for a clear division of labor.
Some team members were responsible for removing the fishing lines that held the cubes together during layer installation, others pushed the optical fibers through the designated holes where the fishing lines had been removed, and some maintained a constant supply of fibers for the fiber pushers.
The fiber insertion process took an additional two months.
Afterward, I returned to focus on my analysis work while the electronics were diligently added and calibrated.
The assembly was finally completed in October 2023 and was ready to be installed in the ND280 pit.
In June 2024, the entire ND280 upgrade was completed and began taking data for the first time.
The data event display images presented in this thesis were collected during the June 2024 run.

As a doctoral student, assembling a detector and witnessing its functionality was a truly unique and fortunate experience.
Although it is unfortunate that there was not enough time to perform a full cross-section measurement using the new data, my Monte Carlo (MC) analysis and the preliminary data-MC comparisons demonstrate the far-reaching physics potential of the upgraded ND280.
The upgraded detector will be capable of measuring the TKI and COM variables with unprecedented precision.
My tuning project, utilizing existing TKI measurements, already shows promising improvements in neutrino interaction modeling.
It is thus exciting to anticipate the advancements in interaction modeling that future TKI and COM measurements of ND280 will bring.
Beyond advancing SM physics, my HNL sensitivity study of the SFGD demonstrates that it possesses comparable HNL detection capabilities to the gaseous TPCs.
Consequently, the entire upgraded ND280 boasts a detection volume nearly twice its previous size.
Combined with the forthcoming beamline upgrade, T2K has the potential to conduct the most sensitive searches for HNLs produced from kaon and pion decays.

New and upgraded experiments, such as JUNO and SBND, around the world have either commenced operations or are about to do so.
Hyper-K and DUNE are also under active construction.
It is optimistic that we will achieve a sufficient understanding of neutrino interactions, enabling next-generation LBL experiments to make precise measurements of $\delta_{\text{CP}}$ within our lifetime.
I hope that my work can contribute to shortening this timeline, even if only slightly, and when that day arrives, I will be gratified.