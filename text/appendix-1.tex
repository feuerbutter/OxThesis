\begin{savequote}[8cm]
\textlatin{Cor animalium, fundamentum e\longs t vitæ, princeps omnium, Microco\longs mi Sol, a quo omnis vegetatio dependet, vigor omnis \& robur emanat.}

The heart of animals is the foundation of their life, the sovereign of everything within them, the sun of their microcosm, that upon which all growth depends, from which all power proceeds.
  \qauthor{--- William Harvey \cite{harvey_exercitatio_1628}}
\end{savequote}

\chapter{\label{app:1-cardiophys}Review of Cardiac Physiology and Electrophysiology}

\minitoc

\section{HNL Input Parameter Calculations}
\label{sec:hnl-input}
    \subsection{Coordinates transformation between different frames}

    \begin{enumerate}
        \item $M_{N}$, mass of the HNL
        \item $\uae$ and $\uam$, the mixing coffeicient of the extended PMNS matrix between the HNL and the electron neutrino and muon neutrino respectively.
        \item XXXX, position of the centre of the detector relative to the centre of the target (?)
        \item XXXX, angle between the beam axis and the longitudinal axis of the detector.
        \item POT parent scaling, the average number of hadrons produced per POT.
    \end{enumerate}
    As the hadrons produced by the T2K beam are predominantly pions and kaons, the $M_{N}$ range that can be explored from hadron decay is between $140\mev$ to $390\mev$. 
    Furthermore, as the decay channel investigated in this work are the two-body decay channels, namely $\pi$+$e$ and $\pi$+$\mu$, the chosen range for $M_{N}$ is from $140\mev$ (250 if mupi only) to $390\mev$ (???).
    
    $\uae$ and $\uam$ are chosen to be equal to $10^{-7}$. They are chosen to be small such that the first order approximation of the proportionality between the number of observed events and the fourth power of the mixing elements are valid.

    In order to specify the relations bewteen the target position, the beam direction and the detector position accurately, three coordiante systems are used.
    The NEAR frame has the origin inside the target with its $y$ axis parallel to the floor of the target hall.
    The NEAR frame is also the frame used by the NEUT generator.
    The BEAM frame is typically rotated from the NEAR frame with the $z$ axis parallel to the beam direction. 
    The USER frame is the coordiante system setup at ND280 for event selection and analysis.

    Taken from official document, the centre of the USER frame is at $(-3.222,-8.146,280.1)$ in the NEAR frame. 
    The BEAM frame is rotated from the NEAR frame downwards around the $x$-axis by $63.44\mrad$. 
    (CHECK how is 63.44 calculated)

    As \code{BeamHNL} aims to be a cross-experiment package, it requires a standardised flux input, the \code{dk2nu} format. 
    The \code{dk2nu} file is a simple flattree root file that needs to contain the minimally sufficient information for \code{BeamHNL} to work.
    As the T2K flux is available in its specific format, a conversion is required. 
    The required \code{dk2nu} variables and the corresponding variables in a T2K flux file is given in Table~\ref{tab:dk2nu-t2k-conv}.
        

\section{Anatomy}
\label{sec:anatomy}


\section{Cellular Electromechanical Coupling}
\label{sec:electromech}
