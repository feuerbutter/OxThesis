\begin{savequote}[8cm]
\textlatin{A deep question is one where either a yes or no answer is interesting.}

  \qauthor{--- Niel Bohr \textit{F. Reienes, Nobel Lecture 1995}}
\end{savequote}

\chapter{\label{ch:intro}Prologue} 

\minitoc

Neutrinos are a type of elementary particle that form our material world. 
They are crucial in addressing one of the most fundamental questions in physics: the existence of our universe. 
Cosmological observations have established that our universe is dynamic, implying that it has evolved into its present state through certain physical processes since its inception.
In nature, physical processes follow certain rules of symmetry, and in particular, Charge-Parity (CP) symmetry dictates that particles and antiparticles should interact in the same way. 
If the universe began with an equal number of particles and antiparticles, and if all physical processes obey CP symmetry, the universe should still contain the same number of particles and antiparticles today.
This presents a problem because particles would annihilate with their antiparticle counterparts to release energy, leading to a universe full of energy but devoid of matter, and hence no stars or planets, let alone life.
Clearly, this cannot be true because we do exist, so either the universe did not begin with the same number of particles and antiparticles, or some physical processes violate CP symmetry such that an excess of particles can accumulate over time. 
While there is no evidence for the former, CP violation is indeed observed in certain processes. 
However, the presently measured extent of CP violation is insufficient to account for the vast amount of matter in our universe, and the only known source of potential CP violation yet to be measured lies in the neutrino sector. 
Hence, a precise measurement of CP violation in neutrinos is a significant step towards answering the matter-antimatter asymmetry mystery and will be a strong indicator of new physics beyond the Standard Model.

CP violation in neutrinos is measured by analysing neutrino oscillations, i.e., the phenomenon that neutrinos can switch between the three known types, namely electron neutrino, muon neutrino, and tau neutrino, as they travel through space. 
Long-baseline (LBL) neutrino experiments are built to measure the difference between neutrino oscillations and antineutrino oscillations to quantify CP violation, for example, the Tokai-to-Kamioka (T2K) experiment. 
The T2K experiment utilises the Super Kamiokande (Super-K) detector as the far detector (FD) to measure the oscillated neutrino spectrum, while the neutrino spectrum near the beam source is measured by a near detector (ND), ND280, located 280 meters from the source. 
It has produced the first constraint on CP violation~\cite{T2K:2019bcf}, but due to limited statistics, the measurement still has a large range with considerable uncertainties.
Super-K will be replaced by Hyper Kamiokande (Hyper-K)~\cite{Hyper-Kamiokande:2018ofw} in 2027, which is much larger than Super-K and will increase statistics many-fold. 
The uncertainties in the measurement of CP violation will be dominated by systematic uncertainties. 
One of the most significant systematic uncertainties lies in the neutrino-nucleus interaction modelling. 
Although the neutrino source beam energy spectrum is well studied, the energy reconstruction of each neutrino is not directly measurable as it leaves no visible track in detectors. 
Thus, we have to rely on approximate neutrino-nucleus interaction models to reconstruct its energy from the interaction products, of which the hadrons and charged leptons could be detected and their energy could be measured if energetic enough. 
Hence, reducing particle energy measurement uncertainty and developing a more sophisticated neutrino-nucleus interaction model are crucial for future CP violation measurements.
The ND280 has been upgraded to achieve this goal, and the research of my thesis centres around this upgrade. 

This thesis is structured as follows: Ch.~\ref{ch:nu-hist} will provide a brief history of neutrino physics, and Ch.~\ref{ch:nu-theo} lays the theoretical foundation necessary for subsequent discussion. 
The following chapter, Ch.~\ref{ch:t2k}, will describe the T2K experiment and its upgrade in detail. 
Ch.~\ref{ch:techniques} elaborates on the development of multiple selections using novel techniques, paving the foundation for physical analyses presented in Ch.~\ref{ch:datamc}. 
As cross-section measurements using the upgraded ND are not possible yet, Ch.~\ref{ch:tuning} presents one important application of cross-section analysis—tuning—using existing measurements.
Besides Standard Model physics, the upgraded ND also holds promising potential for exotic searches.
Ch.~\ref{ch:hnl} presents a sensitivity study for the upgraded ND using the heavy neutral lepton (HNL) simulation package, \code{BeamHNL}, in the \genie~\cite{Andreopoulos:2015wxa} event generator.
The last chapter summarises this thesis and suggests possible extensions of this work.


