\begin{savequote}[8cm]
\textlatin{Neque porro quisquam est qui dolorem ipsum quia dolor sit amet, consectetur, adipisci velit...}

There is no one who loves pain itself, who seeks after it and wants to have it, simply because it is pain...
  \qauthor{--- Cicero's \textit{de Finibus Bonorum et Malorum}}
\end{savequote}

\chapter{\label{ch:7-com}Centre-of-momentum variables} 

\section{Introduction}
Significant efforts have been devoted to measuring Charge-Parity (CP) violation in the neutrino sector through long-baseline (LBL) experiments. 
\blue{In these experiments, CP violation is quantified by the difference between neutrino oscillation and anti-neutrino oscillation. 
LBL experiments quantify $\dcp$ by comparing the energy spectra of $\nue$ and $\nuebar$ oscillated from $\numu$ and $\numubar$, respectively.
There are two major challenges to this measurement. 
Firstly, the far detector of an LBL experiment is by design hundreds of kilometres away from the neutrino source, which unavoidably lead to low statistics.
Secondly, as the energy of each incoming neutrino is unknown and thus the type of the individual neutrino bound-nucleon interaction is also unknown, oscillation predictions have to take the form of energy spectra for a chosen final state topologies.
Accurate $\dcp$ measurements heavily depend on neutrino interaction models estimating contributions of different neutrino-nucleon interactions to final-state topologies~\cite{NuSTEC:2017hzk}.
Large-scale experiments, such as Hyper-Kamiokande~\cite{Hyper-Kamiokande:2018ofw} and the Deep Underground Neutrino Experiment (DUNE)~\cite{DUNE:2015lol,DUNE:2016evb,DUNE:2016hlj,DUNE:2016rla,DUNE:2021tad}, address this challenge by constructing gigantic far detectors to increase event rates.
To match the reduction of statistical uncertainties, it is critical to develop advanced neutrino interaction models or to better constrain existing ones to minimize the systematic uncertainties.}

To better understand the complex neutrino-nucleus interactions, new or upgraded experiments with sophisticated detectors have commenced to explore a larger interaction kinematic phase space and to collect a significantly larger amount of data. 
For instance, the Tokai-to-Kamioka (T2K) experiment~\cite{T2K:2011qtm} has upgraded its near detector (ND) and started data collection in June 2024. 
The Super Fine-Grain Detector (SFGD), part of the T2K ND upgrade~\cite{T2K:2019bbb}, provides improved proton detection with lower thresholds, higher resolution, and greater efficiency. 
Meanwhile, the Short-baseline Near Detector (SBND)~\cite{MicroBooNE:2015bmn} has also begun operations in 2024. 
It is a new LArTPC with an active mass of 112 ton placed at $110$ m from the neutrino source. 
Due to its large active mass and proximity to the source,  it is expected to collect a huge number of neutrino interaction events each year.  
On one hand, the expanded kinematic phase space allows for measuring new variables. 
On the other, the influx of high-quality data offers an ideal testing ground for novel measurement techniques.
Taking advantage of these advancements, it is timely to explore new ideas involving pions in the final states, given the broad energy spectrum of the DUNE beamline, which includes substantial contributions from resonance production comparable to quasi-elastic interactions, 

One effective method of utilizing the near detector data is to constrain model parameters through tuning.
Successful examples~\cite{GENIE:2021zuu,GENIE:2021wox,GENIE:2022qrc} have shown improved data-Monte Carlo (MC) agreement after tuning existing models using various combinations of measurements from different experiments. 
However, the neutrino-nucleus interaction is a convolution of multiple processes: the nucleon initial state (IS), the neutrino-nucleon interaction, and final state interactions (FSI). 
Many variables are affected by all these processes, making it challenging to study the different models in isolation. 
Nuclear effects, such as IS and FSI, occur within the nucleus and remain unobservable with current detectors, making them a major source of systematic uncertainties.
Cleverly constructed variables, such as Transverse Kinematic Imbalance (TKI)~\cite{Lu:2015hea, Lu:2015tcr} or Generalized Kinematic Imbalance (GKI)~\cite{MicroBooNE:2023krv}, are sensitive to nuclear effects, and past measurements have successfully constrained models~\cite{GENIE:2024ufm}. 
While TKI is sensitive to both IS and FSI, except $\dat$, which is predominantly sensitive to FSI but is affected by small uncertainties in the neutrino direction, new variables like $\plong$ \cite{Baudis:2023tma} are designed to be sensitive to specific nuclear effects, such as the removal energy. 

Having more specialized measurements, like $\plong$, can further fine-tune our models, especially in light of the improved detection capabilities. 
This work proposes a new set of variables, called center-of-momentum (COM) variables, for charge current single pion single proton ($\ccopiop$) events, timely for the increasingly precise measurements with pions in the final states
COM variables enable more focused studies of FSI by differentiating between FSI models independently of IS models.

This paper will elaborate on the concept of the COM variables and present MC analysis results focusing on the COM angle and demonstrating its ability to distinguish FSI models and its independence from IS.

\section{The COM Variables}



\minitoc