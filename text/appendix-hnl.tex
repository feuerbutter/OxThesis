\begin{savequote}[8cm]
\textlatin{Cor animalium, fundamentum e\longs t vitæ, princeps omnium, Microco\longs mi Sol, a quo omnis vegetatio dependet, vigor omnis \& robur emanat.}

The heart of animals is the foundation of their life, the sovereign of everything within them, the sun of their microcosm, that upon which all growth depends, from which all power proceeds.
  \qauthor{--- William Harvey \cite{harvey_exercitatio_1628}}
\end{savequote}

\chapter{\label{app:hnl}BeamHNL Technical Details}

\minitoc

\section{\code{dk2nu} conversions}
\label{sec:app-hnl-dk2nu}

    TABLE- Conversion of variables from T2K flux file to \code{dk2nu}.

\section{HNL Input Parameter Calculations}
\label{sec:app-hnl-input}
    This section elaborates on the relevant parameters in the \genie \code{BeamHNL} configuration file, \code{CommonHNL.xml}. 
    The parameters present in the configuration file but not mentioned are set to default values.
    
    \subsection{Common parameters}
    There are several parameters that are common for HNL searches independent of experimental setups under the parameter set, \code{ParameterSpace}. \code{InterestingChannels}.
    \begin{enumerate}
        \item \code{HNL-Mass}, mass of the HNL.
            As the hadrons produced by the T2K beam are predominantly pions and kaons, the $M_{N}$ range that can be explored from hadron decay is between $140\mev$ to $390\mev$. 
            Furthermore, as the decay channel investigated in this work are the two-body decay channels, namely $\pi$+$e$ and $\pi$+$\mu$, the chosen range for $M_{N}$ is from $140\mev$ (250 if mupi only) to $390\mev$.
        \item \code{HNL-LeptonMixing}, the mixing coefficient of the extended PMNS matrix between the HNL and the electron neutrino and muon neutrino respectively.
            $\uae$ and $\uam$ are chosen to be equal to $10^{-7}$. 
            They are chosen to be small such that the first order approximation of the proportionality between the number of observed events and the fourth power of the mixing elements are valid.
        \item \code{HNL-2B\_x\_y}, the switch for a particular HNL decay channel, where x and y are the two decay products, for example \code{mu} and \code{pi}.
            If one wants to investigate all decay channels without having to worry about accounting for the proper decay widths, one could set all the channels to be true, as they account for the vast majority of the decay channels.
    \end{enumerate}

    \subsection{T2K-specific parameters}
    The parameters need to be modified for the T2K ND upgrade are contained in \code{CoordinateXForm} and \code{FluxCalc}.
    The \code{CoordinateXForm} set specifies the geometric relation between the beam and the detector. 
    In order to specify the relations between the target position, the beam direction and the detector position accurately, three coordiante systems are used, namely NEAR, BEAM and USER.
    The NEAR frame has the origin inside the target with its $y$ axis parallel to the floor of the target hall.
    This frame is also the global frame used by the NEUT generator.
    The BEAM frame has the $z$ axis parallel to the beam direction, which is typically rotated from the NEAR frame.
    The USER frame is the coordinate system setup at ND280 for event selection and analysis.
    Parameters describing the frame transformation are as follows.
    \begin{enumerate}
        \item \code{Near2Beam\_R}, the rotation of the BEAM frame with respect to the NEAR. 
        According to the beam centre coordinates given in the T2K flux file, the variables \code{Bxfd} and \code{Byfd}, the beam centre has a $(x,y)$-coordinate, $(0,-1780.400)$~cm at the centre of ND280, which is located at $z=28010$~cm. 
        This correspond to a rotation of $\arctan(-1780.400/28010)=-0.06348$~rad around the $x$-axis.
        Hence, this parameter should be set to \code{0.0 ; 0.0 ; -0.06348}. 
        \item \code{Near2User\_T}, position of the centre of USER in the NEAR frame.
        For T2K, the ND centre is located at \code{(-3.222 ; -8.146 ; 280.1)}~m.
        \item \code{DetCentre\_User}, position of the detector centre ikn the USER frame. 
        As origin of the USER frame is the centre of the whole ND280, the centre of the upgrade part, namely SFGD and HAT, is located at \code{(0.0 ; 0.029985 ; -1.94394)} in the USER frame.
    \end{enumerate}   

    Moreover, the \code{FluxCalc} set contains additional flux-related specifications:
    \begin{enumerate}
        \item \code{ParentPOTScalings}, the average number of HNL parents produced per POT for all, muons, pions and $K^{L}$.
        \item \code{IsParentOnAxis}, this should be set to \code{false}, as ND280 is located off axis.
        \item \code{InputFluxesInBEAM}, this should be set to \code{false}, as the T2K flux input is given in the NEUT frame, i.e. the NEAR frame, instead of the BEAM frame.
        \item \code{CollectionRadius}, the radius of the detector in the USER frame, which is set to \code{6.0}~m. 
        It should be large enough to cover the whole ND280.
    \end{enumerate}



