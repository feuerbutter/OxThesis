\begin{savequote}[8cm]
Leider kann ich nicht persoenlich in Tuebingen ercscheinen, \\
da ich infolge eines in der Nacht vom 6. zum 7 Des. in Zuerich stattfindenden Balles hier unabkoemmlich bin.

  Unfortunately, I cannot personally appear in Tübingen since I am indispensable here in Zürich \\ because of a ball on the night from December 6 to 7.
  \qauthor{--- Wolfgang Pauli~\cite{Pauli:1930pc}, translation: Kurt Riesselmann}

\end{savequote}

\chapter{\label{ch:nu-hist}The History of Neutrinos}
\minitoc

The field of neutrino physics is relatively young and has developed rapidly over the years.
This chapter will go back in time to the birth of neutrino physics and note some of the important milestones.
As this only serves as a prologue to the thesis, it is by no means a complete historical account.
Let us dive into the fruitful history of the field.

\section{The Past}
The field of neutrino physics has a dramatic opening that many know of—in a letter~\cite{Pauli:1930pc} from Wolfgang Pauli to the Radioactivity Conference in 1930, which he did not attend in person as he was indispensable at a ball in Zürich. 
The letter proposed a solution to the then very puzzling problem of $\beta$ decay.
It was thought of as a two-body decay, where the heavier nucleus decays to a lighter nucleus and an electron, which carries away a well-defined amount of energy due to the mass difference between the parent and daughter nucleus.  
However, the measured electron energy spectrum was a wide distribution rather than the fixed value expected. 
In his letter, Pauli postulated a third, but undetected, particle in the decay that shared the energy budget with the electron so that the sum of the third particle's energy and the electron's energy is equal to the fixed value expected, but not the electron energy alone. 
Originally, he coined this particle the ``neutron'', but this name did not stick and was given to the now-known neutron discovered by James Chadwick in 1932.
The present name, ``neutrino'', was later coined by Edoardo Amaldi in a conversation with Enrico Fermi, who later published the seminal paper on his theory of beta decay in 1934~\cite{Fermi:1934hr}, laying the theoretical foundation for subsequent experimental work.
However, the cross-section for neutrino interaction is estimated to be so small, making it extremely difficult to detect.
Even Pauli lamented, ``I have done a terrible thing. I have postulated a particle that cannot be detected.''

It was such a formidable task that Frederick Reines even proposed using the atomic bomb as the neutrino source to have an intense source of neutrinos for detection.
Fortunately, Reines and his collaborator, Clyde Cowan, discovered that the delay between the signal of neutron capture and the signal of annihilation between electron and positron, where the neutron and the position are the products of the antineutrino interactiing with a proton, is distinct and could be utilised to significantly reduce backgrounds, allowing a less intense (and less dangerous) source of antineutrinos to be used.
It was not until about 26 years after the postulation of the neutrino that it was directly detected in the Cowan-Reines experiment at the Savannah River Plant in 1956~\cite{Cowan:1956rrn}. 
The detector was mainly composed of $1400$ litres of liquid scintillator and $200$ litres of cadmium-loaded water as target and neutron capturer.
This remarkable achievement was recognised with a Nobel Prize in 1995.

After the Cowan-Reines experiment, the existence of the neutrino was beyond doubt, and the field of neutrino physics developed rapidly. 
There were two naturally complementary directions for development.
One was to build larger and more advanced detectors to measure the existing neutrino sources, for example, solar neutrinos, reactor neutrinos, atmospheric neutrinos, and others.
The other was to include neutrino production as part of the experiment, which became the field of accelerator neutrinos.

Roughly about the same time, Raymond Davis was also working at the Savannah River Plant, trying to use the inverse beta decay of chlorine to argon, suggested by Pontecorvo~\cite{Pontecorvo:1946mv}, to detect neutrinos.
His experiment measured at least a $20$ times smaller cross-section for antineutrinos than for neutrinos in the inverse beta decay turning chlorine into argon~\cite{Davis:1959pba}.

Later, he applied the same technique to measure the solar neutrino flux with a detector of the size of $378,000$ litres at the Homestake Mine~\cite{Davis:1964zz}, sufficient to detect solar neutrinos based on the Standard Solar Model calculation done by John Bahcall~\cite{Bahcall:1964ya}.
It was to everyone's surprise that the Homestake experiment only observed about one-third of the expected flux~\cite{Davis:1968cp}, a discrepancy known as the solar neutrino problem.
Davis and Bahcall did meticulous checks on their results and found no errors large enough to explain the discrepancy.

This phenomenon sparked many efforts, both experimentally and theoretically, to resolve the observed discrepancy.
The idea of neutrino oscillation was actually proposed even before the solar neutrino problem by Pontecorvo in 1957~\cite{Pontecorvo:1957qd} and completed by Maki, Nakagawa, and Sakata in 1962~\cite{Maki:1962mu}.
However, it was just one of the possible solutions back then among many other theories.
Papers on the other piece of the solution, the Mikheyev–Smirnov–Wolfenstein (MSW) matter effect~\cite{Wolfenstein:1977ue,Mikheyev:1985zog}, were published in the 1980s.
It is the phenomenon that the neutrino propagation is affected by matter density. More details are provided in the next chapter.
The combined solution, the large-mixing-angle MSW solution, provided an attractive answer to the solar neutrino problem.
Subsequent experiments, such as Kamiokande, a $3,000,000$-litre water Cherenkov experiment led by Masatoshi Koshiba~\cite{Kamiokande-II:1989hkh}, GALLEX~\cite{GALLEX:1998kcz}, and SAGE~\cite{SAGE:1999nng}, confirmed the solar neutrino deficit.
Another profound discovery at that time was the first measurement of supernova neutrino, from SN1987A, by the Kamiokande collaboration~\cite{Kamiokande-II:1987idp}.
The definitive evidence for solar neutrino oscillation came in 2001, when the Sudbury Neutrino Observatory (SNO) experiment, with a $1,000,000$-litre heavy water Cherenkov detector, published its result~\cite{SNO:2001kpb}, directly showing that the observed deficit of electron neutrinos was due to oscillation into other flavours.
The following year, Davis and Koshiba received a Nobel Prize in physics.
Meanwhile, in 1998, the upgrade of Kamiokande, the Super-Kamiokande (Super-K) ($50,000$-tonne water Cherenkov detector), reported the observation of neutrino oscillation in atmospheric neutrinos~\cite{Super-Kamiokande:1998kpq}.
Takaaki Kajita (Super-K) and Arthur McDonald (SNO) shared a Nobel Prize for their work on neutrino oscillation in 2015.

Along the other development route, Pontecorvo suggested the possibility of using accelerated protons hitting targets to produce mesons that decay to produce a beam of neutrinos in 1959~\cite{Pontecorvo:1959sn}.
A year later, a similar proposal was made independently by Mark Schwartz~\cite{Schwartz:1960hg} to use accelerator-produced neutrinos to address another puzzle at that time, whether the muon neutrino is the same as the electron neutrino.
Schwartz and his collaborators, which included Leon Lederman and Jack Steinberger, used the Alternating Gradient Synchrotron at the Brookhaven National Laboratory to produce the first accelerator neutrino beam.
The neutrino interaction was recorded by a $10$-tonne spark chamber, which was able to reconstruct clear muon tracks from the interaction.
In 1962, the collected data proved that muon neutrinos are distinct from electron neutrinos~\cite{Danby:1962nd}.
Schwartz, Lederman, and Steinberger were awarded the Nobel Prize in 1988.

For the subsequent decades, accelerator neutrinos were used to help progress the more general field of particle physics due to the unique nature of neutrino scattering.
For instance, neutral current interaction was first observed by the Gargamelle bubble chamber at CERN using a beam of neutrinos~\cite{GargamelleNeutrino:1973jyy}.
High-energy neutrino beams were also deployed to test the electroweak theory and study the nucleon structure by the CDHS experiment at CERN~\cite{Schlatter:2015nxk}.

Duing the 1990s, in light of the evidence of neutrino oscillations observed in solar neutrinos and atmospheric neutrinos, accelerator neutrinos were first proposed to be used to study neutrino oscillations in two long-baseline (LBL) experiments, namely the K2K experiment in Japan~\cite{Wilkes:1997hm} and the Main injector neutrino oscillation search (MINOS) experiment in the U.S.~\cite{MINOS:1995txm}.
Both of these LBL experiments deployed two detectors, similar in design, a small one near the neutrino source and a large one far away, to compare the neutrino fluxes at the two locations to measure the oscillation.
The K2K experiment used Super-K as the far detector at approximately $250$ km away, while MINOS uses a 6,000-tonne steel detector at $735$ km away.
Their results~\cite{K2K:2002icj,MINOS:2006foh} in 2000s after a few years of data collection corroborated the Super-K results, confirming the oscillation of muon neutrinos into other flavours.

The field has entered a phase of precision measurement that will be illustrated in the next subsection.
The two complementary development routes in the past have combined in long-baseline experiments to elucidate the few remaining questions in the field.

% To better appreciate the historical progress of neutrino physics, it is clearer to put it in the perspective of the overall development of particle physics.

% Electron was discovered by J.J. Thomson in 1897.
% Proton was discovered by Rutherford in 1917.
% Neutron was discovered by Chadwick in 1932.
% Paul Dirac published the famous Dirac equation in 1928, predicting the existence of the positron, the anti-particle of the electron, which was discovered by Carl Anderson in 1932.
% Beta decay was discovered by Lise Meitner and Otto Hahn in 1938 to be a weak interaction process, where a neutron decays to a proton, an electron and an anti-neutrino.
% The muon was discovered by Carl Anderson in 1936, which was initially thought to be the Yukawa particle responsible for the strong nuclear force.

% ``Weak interaction'' is first coined by Enrico Fermi in 1933 to describe the force responsible for beta decay. <CHECK>
% On the theoretical front, Fermi proposed a theoretical solution to the beta decay in 1934.

% In 1954, CN Yang and Robert Mills published the seminal paper on non-abelian gauge theory, laying the foundation for the Standard Model.
% In 1960s and 1970s, rapid progress was made in the unifications of the electromagnetic and weak interactions, leading to the Glashow-Salam-Weinberg (GSW) model, who shared the Nobel Prize in Physics in 1979. 
% The discovery of the W and Z bosons at CERN in 1983.

% It was suggested by Markov [65], Pontecorvo [66], and Schwartz [67] to use proton accelerators to produce high energy
% neutrino beam from pion decays to perform experiments like:
% nu+n->mu+n
% etc
% The experiments performed at the Brookhaven National Laboratory (BNL) by Danby et al. [71] and later at
% CERN by Bienlein et al. [72] 
% observed only muon but never elec tron, confirming numu and nue are different (1962)

% 1965 BNL introducted the POT idea as a measure of the neutrino flux

% Tau in 1975
% the existence of a new flavor of
% neutrinos ντ was proposed, which was observed much later in the DONUT experiment [74,75] in 2000 at the Fermilab

% It was Fermi [3,4] and Perrin [5] who first discussed the determination of the neutrino mass from the study of the end-point spectrum of beta decay.

\section{The Present}
With neutrino oscillation finally confirmed, the field of neutrino physics has a fruitful past and a clear theoretical framework laid out.
One present task for the field is the precise measurement of the parameters in the theoretical framework, namely the neutrino masses and elements of the Pontecorvo–Maki–Nakagawa–Sakata (PMNS) matrix~\cite{Pontecorvo:1957qd, Maki:1962mu}.
The PMNS matrix describes the mixing between the three mass eigenstates of neutrinos to form the three flavour eigenstates, namely electron neutrino, muon neutrino, and tau neutrino.
It can be parameterised by three mixing angles, namely $\totwo$, $\tothr$ and $\ttthr$, and a CP violation phase, $\dcp$.
The specific meaning of the parameters in the PMNS matrix will be illustrated later in Sec.~\ref{sec:oscillation}, while the respective measurements are first given here without breaking the flow.
% The measurements given in this section follow from the latest Particle Data Group release~\cite{ParticleDataGroup:2024cfk}.

The mass of the electron neutrino can be studied by investigating the shape of the high-energy tail of the electron spectrum in tritium beta decay, a method proposed in Fermi's theory paper on $\beta$-decay~\cite{Fermi:1934hr} and independently by Francis Perrin~\cite{Perrin1933}.
The KATRIN experiment employs this method and produces one of the most stringent upper bounds, $0.8~\ev$ with $90\%$ confidence level~\cite{KATRIN:2021uub}.
There are other methods to extract the neutrino mass, such as from cosmological observations~\cite{Brieden:2022lsd}, but as this thesis focuses on neutrino oscillation, these will not be elaborated for conciseness.

As neutrino oscillation depends on the mass difference squared as well, oscillation experiments are able to measure the neutrino masses indirectly through the mass differences in addition to the PMNS matrix elements.
Contrary to the electron neutrino mass measured from $\beta$-decay, the mass differences measured in oscillation experiments are the differences between the neutrino mass eigenstates.
The general structure of the mass eigenstates is that there is a small difference between $m_2$ and $m_1$, but $m_3$ is either much smaller or larger than $m_2$ and $m_1$.
The relation between $m_3$ and the others is the mass hierarchy problem.
The case where $m_3$ is the largest is called the Normal Ordering (NO), and the case where $m_3$ is the smallest is the Inverted Ordering (IO).

In 2013, the KamLAND collaboration made one of the most precise measurement of $\delmtwoo$, $\delmtwoo=7.53\pm0.18~\times 10^{-5}~\ev^2$~\cite{KamLAND:2013rgu}.
Note that this difference is not the absolute value as it is known that $m_2>m_1$.
As for $\delmthrt$, the estimation is produced from a global fit of accelerator, reactor, and atmospheric data by PDG~\cite{ParticleDataGroup:2024cfk}, giving $\delmthrt=2.455\pm0.028~\times 10^{-3}~\ev^2$ (NO) and $\delmthrt=-2.529\pm0.029~\times 10^{-3}~\ev^2$ (IO).

As for the mixing angles, the current best estimate of $\totwo$ is given by a fit of the KamLAND measurement and solar measurements in 2016, producing $\sin^2(\totwo)=0.307^{+0.013}_{-0.012}$.
The estimation of $\ttthr$ is done by the PDG using accelerator and atmospheric neutrino data~\cite{ParticleDataGroup:2024cfk}, leading to $\sin^2(\ttthr)=0.553^{+0.016}_{-0.024}$ (IO) and $\sin^2(\ttthr)=0.558^{+0.015}_{-0.021}$ (NO).
The last mixing angle, the average of $\tothr$, is obtained by PDG using accelerator and reactor data, giving $\sin^2(\tothr)=2.19\pm0.07$.
No definitive measurement of $\dcp$ has been made yet. 
The T2K experiment has produced the first constraint on $\dcp$~\cite{T2K:2019bcf}, which is $-3.41<\dcp<-0.03$ (NO) or $-2.53<\dcp<-0.32$ (IO) in radians at $3\sigma$ confidence interval.

From being completely unknown to precisely measured, neutrino oscillation is now a well-understood phenomenon except for the mass ordering and the extent of CP violation, which is the pressing goal of the present and future LBL neutrino experiments.
 
\section{The Plans}
To elucidate the remaining questions in neutrino physics, new experiments are planned or under construction and existing experiments are undergoing upgrades.
This sub-section will only briefly describe some of them and it is not meant to be an exhaustive review.

The successor of the Daya Bay experiment, which produced one of the most precise measurements of $\tothr$, the Jiangmen Underground Neutrino Observatory (JUNO), is set to start operation in summer 2025~\cite{ScienceNews2025}.
Located near the Yangjiang and Taishan nuclear power plants, JUNO has a $20,000$-tonne liquid scintillator.
It is projected to determine the mass ordering with $3~(3.1)\sigma$ significance for NO (IO) in approximately $7$ years of data taking~\cite{Paoloni:2024atc}.

The successor of the Super-K experiment, the Hyper-Kamiokande (Hyper-K)~\cite{Hyper-Kamiokande:2018ofw}, is under construction and is expected to operate in 2027.
Hyper-K is a $258,000$-ton water Cherenkov detector, about $8$ times the size of its predecessor, Super-K.
It uses the same beamline as the T2K experiment and measures the electron neutrino appearance to determine $\dcp$.
With $10$ years of data taking, it is projected to exclude CP conservation to $5\sigma$ for more than $60\%$ of true $\dcp$ values and to $3\sigma$ for $75\%$ of $\delta_{CP}$ values assuming the systematic uncertainties are controlled at $2.7\%$~\cite{Jesus-Valls:2024ady}.

Another large-scale LBL neutrino experiment is the Deep Underground Neutrino Experiment (DUNE)~\cite{DUNE:2016hlj,DUNE:2015lol,DUNE:2016evb,DUNE:2016rla,DUNE:2021tad}.
Its far detector consists of four $17,000$-ton Liquid Argon Time Projection Chambers (LArTPC) and it has a beamline of $1300$~km, which enables it to be highly sensitive to the mass ordering as well.
With the novel LArTPC technology, it is expected to reach $5\sigma$ for the mass ordering with 3 years of data taking and reach $3\sigma$ for $75~\%$ of $\delta_{CP}$ values with about 15 years of data taking~\cite{Gil-Botella:2024duf}.
There is no official date for DUNE yet, but the test experiment for its far detector, the ProtoDUNE experiment, has operated for more than 2 years at CERN.

In the projected sensitivity of these future experiments, the importance of reducing systematic uncertainties cannot be over-emphasised as results obtainable with years of data taking would need decades instead if the systematic uncertainties do not reach the required level.
This is exactly why T2K is upgrading its near detector to better understand neutrino interactions to reduce systematics, which is exactly what this thesis is about.

Besides the measurement of the known parameters, there are also a plethora of beyond the Standard Model searches both at existing and future experiments.
Moreover, there has been impressive progress on utilising neutrinos for cosmological observation by experiments, such as IceCube~\cite{IceCube:2023ame}.
Although they are not elaborated in this thesis, they represent the highly exciting development of the field.
From the $1400$-litre liquid scintillator detector in the Cowan-Reines experiment to the $1,000,000$-litre heavy water Cherenkov detector of SNO and to the $258,000,000$-litre water Cherenkov detector of Hyper-K, the field of neutrino physics has grown rapidly.
From being the most elusive particle, which was thought never possible to detect, to a sophisticated understanding of neutrino oscillation, our understanding of the properties of neutrinos has improved drastically.
With the incredible experiments in plan and running, it is promising that a definitive measurement of all the neutrino properties is reachable, and its results will be profound and shed light on new physics directions.
