\begin{savequote}[8cm]
\textlatin{Neque porro quisquam est qui dolorem ipsum quia dolor sit amet, consectetur, adipisci velit...}

There is no one who loves pain itself, who seeks after it and wants to have it, simply because it is pain...
  \qauthor{--- Cicero's \textit{de Finibus Bonorum et Malorum}}
\end{savequote}

\chapter{\label{ch:3-t2k}The T2K Experiment} 

\minitoc

It is a long-baseline neutrino experiment, measuring neutrino oscillation.

More specifically, $\dcp$ can be measured in long-baseline neutrino experiments, for example, the Tokai-to-Kamioka (T2K) experiment\cite{T2KEXP}. 
T2K is a long-baseline neutrino experiment located in Japan. 
The neutrino source is generated in J-PARC in Tokai. 
A near detector, ND280, is placed at $280\mathrm{m}$ from the generation point, and a far detector, the Super Kamiokande (SK), is situated 296km away. As neutrinos interact extremely weakly, direct measurement is not yet possible. These detectors measure the neutrino energy spectra from the product particles from the interaction between neutrinos and hydrocarbons in ND280 and water molecules in SK. The difference between neutrino oscillation and anti-neutrino oscillation can be measured by comparing the neutrino energy spectra observed at the far and near detectors in the neutrino mode and the anti-neutrino mode. In 2020, the Tokai-to-Kamioka (T2K) experiment\cite{T2KEXP} has made the first measurement of $\dcp$\cite{T2Knature}, which ruled out CP conservation at the $95\%$ confidence level. It is an impressive first step, but it is still limited both by statistical and systematic uncertainties. 

 