Lord Kelvin identified two “clouds” over 19th-century physics in 1900, which were resolved relatively promptly, yet additional “clouds” have since emerged and dissipated.
One of the most prominent remaining ``clouds'' is Charge-Parity violation (CPV) in the lepton sector.
The T2K experiment is dedicated to measure CPV via neutrino oscillation.
Although significant progress has been made, its potential is limited by an incomplete understanding of neutrino–matter interactions.
To address this shortfall, the T2K near detector (ND) has been upgraded, and I had the privilege of participating in its assembly and research.
This thesis focuses on the new techniques and physics potential enabled by the ND upgrade.
Specifically, I have adopted the Elastically Scattered and Contained proton selection and developed the pion trackless reconstruction.
These novel techniques are illustrated with Monte Carlo (MC) studies and the first glimpse of a preliminary comparison between MC and data collected by the upgraded ND.
The upgrade also enables measurements of Transverse Kinematic Imbalance (TKI) variables with unprecedented precision. 
Furthermore, a new set of centre-of-momentum (COM) variables has been devised to uniquely probe final-state interactions and to provide a complementary handle to TKI for neutrino–hydrogen event selection.
Their power is demonstrated in a tuning project where existing TKI data are used to constrain the current model, significantly improving the agreement between MC and data.
Finally, this thesis concludes with a sensitivity estimation for a heavy neutral lepton search at the upgraded ND, exploring the elusive “clouds” in an ever clearer sky of physics.