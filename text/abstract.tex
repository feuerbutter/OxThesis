As the two ``clouds'' for 19th century physics identified by Lord Kelvin in 1900 were eventually resolved relatively promptly, more ``clouds'' had appeared and dissolved.
One of the greatest remaining ``clouds'' in the sky of physics is the Charge-Parity violation (CPV) in the lepton sector.
The T2K experiment is dedicated to measure CPV through neutrino oscillation.
It has made significant progress, but its potential is limited by the lack of an excellent understanding of the interaction between neutrino and matter.
To address this shortfall, the near detector (ND) of T2K has undergone an upgrade, and I was privileged to be part of its assembly team.
This thesis centres around the new techniques and the physics potential of the T2K ND upgrade.
More specifically, I have adopted the Elastically Scattered and Contained proton selection and invented the pion trackless reconstruction.
These novel techniques will be illustrated with Monte Carlo (MC) studies and the very first glimpse of the data collected by the upgraded ND.
The physics potential of the upgrade is exemplified with capability to measure Transverse Kinematic Imbalance (TKI) variables to an unprecedented precision.
Furthermore, a set of new variables, the centre-of-momentum (COM) variables, are cleverly devised to offer unique probes for final-state interactions and a new and complementary to TKI handle for the neutrino-hydrogen event selection.
The power of these high-level variables is demonstrated with a tuning project, where using the existing TKI data to constrain the current model leads to significantly better agreement between MC and data.
Finally, this thesis concludes with the sensitivity estimation of a heavy neutral lepton search at the upgraded ND, looking for the possibly elusive ``clouds'' in the clear sky of physics.